\documentclass[11pt, letterpaper, titlepage]{article}
\usepackage[utf8]{inputenc}
\usepackage[export]{adjustbox}
\usepackage{geometry}
 \geometry{
 a4paper,
 total={168mm,257mm},
 left=20mm,
 top=15mm,
 includefoot,includehead
 }
\usepackage[backend=biber, style=authoryear, giveninits=true, maxbibnames=25, uniquename=init, maxcitenames=2, hyperref=true, dashed=false]{biblatex}			% Benutze Biber/BibLaTeX zum Zitieren
\addbibresource{main.bib}
\usepackage{caption}
\usepackage{subcaption}
\usepackage{graphicx}
\usepackage{svg}
\usepackage{placeins}
\usepackage[hidelinks]{hyperref}
\usepackage{amsmath}
\usepackage[headsepline]{scrlayer-scrpage}
\usepackage{acronym}

\clearpairofpagestyles %Seitenzahl nicht in der Kopfzeile

\title{MeetEU Project - Team Heidelberg - Team 1 -- \\ Identification and Enhancement of novel Sars-CoV-2 NSP13 Helicase Inhibitors}
\author{Linda Blaier, Paul Brunner, Selina Ernst, Valerie Segatz, and Chlo\'{e} Weiler}
\date{February 2024}

\begin{document}

\maketitle

\ihead{\headmark}
\automark{section}  %Kopfzeile gleich dem Sektiontitel
\cfoot{\pagemark}   %\ofood Seitenzahl rechts

\section{Abstract}
Even though the development of vaccines against Sars-CoV-2 was successful during the recent pandemic, the amount of FDA approved drugs for the therapy of Covid-19 is still limited to Paxlovid and Veklury, Olumiant and Actemra \cite{FDA_COVID}. One possibility to accelerate the development of new therapies for Covid19 is to screen already approved drugs for effects against the viral reproduction. In this years MeetEU project, we investigated the NSP13 helicase of Sars-CoV-2 and tried to find compounds that could be repurposed for this therapy, as well as novel compounds that could lead to an effective treatment of Covid19. Using our \textit{in-silico} pipeline enables us to evaluate possible drug candidates, suggest novel structures based on already approved drugs and investigate their toxicity, while being cheaper and less labor intensive than projects limited to wet-lab work. 
HIER KOMMT EINE ABBILDUNG HIN

\newpage
%    Abkürzungsverzeichnis
{\setlength{\parskip}{0.2cm}
\section*{Abbreviations}
    \begin{acronym}[LC-MS/MS23]
        % A B C D E F G H I J K L M N O P Q R S T U V W X Y Z        
        % Abkürzungen
        \acro{MD}{Molecular dynamics}
        \acro{NSP13}{Non-structural protein 13}
        \acro{RTC}{Replication transcription complex}
        \acro{SAscore}{Synthetic accessibility score}
        \acro{ssRNA}{Single-stranded RNA}
        \acro{ZBD}{Zinc binding domain}
        \acro{SARS-CoV-2}{severe acute respiratory syndrome corona virus 2}
       
        % Formelzeichen
        
        
        % als benutzt markierte Acronyme    
        
        
    \end{acronym}
}
\newpage


\section{Introduction}
Even though the development of vaccines against SARS-CoV-2 was successful during the recent pandemic, the amount of FDA approuved drugs for the therapy of COVID-19 is still limited to Paxlovid, Veklury, Olumiant, and Actemra (\cite{FDA_COVID}). Improving the landscape of drugs available for treating COVID-19 would be particularly beneficial for people who are at risk of severe illness, as vaccines may not fully prevent infections. The goal of this year's Meet-EU project is to develop a pipeline to identify possible inhibtiors against the SARS-CoV-2 helicase also known as \ac{NSP13}. There are two main reasons as to why this protein is a promising drug target. For one, it is highly conserved among corona viruses, which means that the virus is unlikely to develop resistances against drugs targeting \ac{NSP13} through rapid mutations in the viral genome (\cite{Spratt_2021}). On the other hand, \ac{NSP13} together with other non-structural proteins forms the \ac{RTC}, which is essential for viral RNA synthesis (\cite{Malone_2022}). Therefore, inhibiting \ac{NSP13} would severely hinder the spread of the virus inside the host. 
The protein consists of five domains, namely the \ac{ZBD}, the stalk domain, as well as 1A, 2A and 1B. The latter three make up the catalytic centre of the protein, where RNA and ATP bind (\cite{NSP13_basics}). \\
Computer-aided structure-based drug discovery can be followed to identify possible inhibitors of the \ac{NSP13} helicase. These can then be further investigated in wet-lab settings. This process involves several steps: (1) Identification of possible binding sites, (2) high-trhoughput screening of ligands for how well they bind the respective pocket, followed by (3) the evaluation of the binding pathways, the kinetics, and thermodynamics (\cite{Sledz_2018}). Hereby, focussing this screening on well documented or already FDA approved compounds is very attractive, as this drug repurposing potentially shortents the development period and therefore also the development costs (\cite{Pushpakom_2019}).
\ac{ssRNA}, \ac{SARS-CoV-2}
\subsection*{}
\subsection{Identification of Consensus Binding Pocket}
In drug discovery, the initial step is to investigate the protein structure in order to analyse potential binding sites. These are cavities on the surface or interior of the protein with suitable properties to bind a ligand. The functionality of a binding pocket is determined by its shape and location, but also by the amino acid residues which define its pyhsicochemical characteristics (\cite{Stank_2016}). 
Different experimental and theoretical procedures exist to analyse the druggability of such binding pockets. In this work, we combined three different \textit{in silico} tools, each following a different algorithm. Fpocket (\cite{package_Fpocket}) utilises a geometry-based algorithm based on Voronoi tesselation and sequential clustering to determine potential binding sites. We also implemented P2Rank (\cite{package_P2Rank, package_PrankWeb, package_PrankWeb3}), which is based on a machine-learning algorithm. 
P2Rank assigns structural, physicochemical, and evolutionary features to points on the solvent-accessible surface of a protein. From this information, the machine-learning model is built and used to predict and rank potential ligand binding sites. Lastly, FTMAP (\cite{package_FTMAP}) was used to validate the binding pocket found with the previously mentioned approaches. FTMAP uses docking results of sixteen small molecules differing in polarity, shape, and size to identify binding hot spots with a fast Fourier transform correlation. The most favourable docked confirmations are determined through energy minimisation and clustering processes.
Finally, the results of all three tools were combined to identify a consensus binding pocket of the NSP13 helicase. The resulting coordinates of the consensus binding pocket were then used for molecular docking simulations. 

\subsection{Molecular Docking}
Molecular Docking programs are used to evaluate binding affinities between a potential drug candidate and the target protein. A key aspect of this task is the prediction of the ligand position, orientation, and conformation. Search-based methods approach this task by continuously modifying the ligand pose, while estimating its quality or likelihood (score) and stochastically trying to infer the global optimum of the scoring function. Among the most widely used tools are AutoDock Vina (\cite{Trott.2010}) and Glide (\cite{Halgren.2004}), which mainly differ in their scoring functions. However, such search-based methods are computationally expensive. Therefore, in order to be able to screen large datasets, search-based methods are generally restricted to a previously defined binding pocket (\cite{Corso.2022}). Consequently, potential other binding sites of a ligand are not assessed. Machine learning-based blind docking approaches try to address that problem by stochastically predicting binding pocket and ligand pose based on learned characteristics and aligning them. The most promising results are achieved by using Diffdock (\cite{Corso.2022}), a generative model which applies a reverse diffusion process to the docking paradigm. In this manner, Diffdock iteratively transforms an uninformed noisy distribution over ligand poses defined by the degrees of freedom involved in docking (position, turns around its centre of mass, and twists of torsion angles) into a learned model distribution (\cite{Corso.2022}). Corso \textit{et al.} thereby describe this process as a progressive refinement of random ligand poses via updates of their translations, rotations and torsion angles.

%\subsection{Lead Drug Enhancement}
%In order to enhance the binding affinity of our drug candidates and thus their performance, we used AutoGrow4 (Version 4.0.3) \cite{package_Autogrow4} to generate novel compounds. Starting with the best binding compounds of our initial docking simulation with AutoDock Vina as generation zero, multiple new structures are generated by combining sub-structures of the first generation or by passing them through a set of possible chemical reactions after converting them into their respective SMILES codes. All of the generated compounds are ranked by their binding affinity. After passing several filters, the best-performing compounds are used as the seed for the next generation. Using this algorithm, compounds are found, which show higher binding affinities than the first generation. As AutoGrow4 labels all new structures by the path by which they were obtained, we can also evaluate the synthesizability.  

\subsection{Estimation of Toxicity and Synthetic Accessibility}
 After identifying the lead compounds that exhibit optimal binding affinity within the consensus pocket, an evaluation of the general toxicity and synthetic accessibility of these compounds was performed using the latest version of the machine-learning tool \textit{e}ToxPred (\cite{pu2019toxpred}). This additional step helps estimate the suitability of the compounds as real-life pharmaceuticals against COVID-19. The Tox-score allows for a general assessment of the predicted risk vs. benefit ratio of the potential NSP13 inhibitors. Moreover, \textit{e}ToxPred allows for an insight into the ease of synthesis, indicated by the \ac{SAscore}. This score reflects the ease and efficiency of producing the molecules in large quantities and consequently their feasibility as potential drugs. 

\subsection{Molecular Dynamics Simulation}
As the last step of our pipeline, a \ac{MD} simulation is conducted using the best-scoring compound as a ligand in the binding pocket of the NSP13 protein. Using GROMACS (Version 2023.3) (\cite{package_GROMACS}), this enables us to interpret the stability of the protein-ligand interaction, as well as to identify important residues for the interaction. Using a given force-field, a set of equations describing different forces between the atoms and residues in the protein and ligand, the movement of all atoms in the system can be simulated and analysed. However, this is only possible in a very limited timeframe with a small time step size. As this process is rather resource-heavy, it has to be conducted on a cluster with access to a GPU. 


\section{Material and Methods}
\subsection{Toxicity and Synthetic Accessibility Prediction using \textit{e}ToxPred}
The general toxicity and synthetic accessibility of the given compounds was estimated using the machine-learning tool \textit{e}ToxPred (\cite{pu2019toxpred}). The SMILES files of the Top100 compounds from AutoDock Vina (\cite{Trott.2010}) served as input for the pre-trained model.The toxicity predictor was pre-trained on the FDA-approuved and the KEGG-drug datasets whose compounds were considered non-toxic as well as the TOXNET and the T3DB datasets whose compounds were considered toxic using a deep-belief-network based model. This predictor yields a Tox-score between 0 and 1 and in accordance to the paper, all compounds with a Tox-score below 0.58 were deemed non-toxic. The synthetic accessibility was reflected in a synthetic accessibility score (SAscore) which was obtained by training an extra-trees-based classifier on NuBBE, UNPD, FDA-approuved, and DUD-E-active datasets.

\section{Results}
\subsection{Top 100 Compounds exhibit low toxicity and high synthetic accessibility}
After identifying the top 100 best-binding ligands through AutoDock Vina, our subsequent analysis focused on evaluating their practical applicability as potential drugs by considering their predicted toxicity and synthetic accessibility. The resulting SAscore and Tox-Score for each compound were visualised in a scatter plot as seen in \ref{eToxPred}. For the visualisation we chose to present 

\begin{figure}[h]
    \begin{center}
      \includegraphics[width=0.9\textwidth]{etoxpred_result.pdf}
    \end{center}
    \caption{\textbf{Scatter plot of predicted toxicity and synthetic accessibility of the 100 best-binding compounds.} The predicted \ac{SAscore} was plotted against the Tox-Score for the top 100 best scorers from AutoDock Vina. The top 3 scorers from Glide are marked in red. The vertical gray line represents the threshold for the toxicity.}\label{eToxPred}
  \end{figure}

Of those 100 compounds, 99 presented with a Tox-score below the threshold of toxicity, meaning that they are in probability not toxic to humans. The median SAscore across all compounds was 2.69. The top scorer from Glide, ZINC000096077632, was predicted to have a Tox-Score of 0.14 and an SAscore of 5.02.

\FloatBarrier

\section{Discussion and Outlook}

\section{Supplementary Material}

\pagebreak
\FloatBarrier
\renewcommand{\bibname}{References}  % damit Literatuverzeicnis mit "References" betitelt
\printbibliography



\end{document}
