\documentclass[11pt, letterpaper, titlepage]{article}
\usepackage[utf8]{inputenc}
\usepackage[export]{adjustbox}
\usepackage{geometry}
 \geometry{
 a4paper,
 total={168mm,257mm},
 left=20mm,
 top=15mm,
 includefoot,includehead
 }



\usepackage[backend=biber, style=authoryear, giveninits=true, maxbibnames=25, uniquename=init, maxcitenames=2, hyperref=true, dashed=false]{biblatex}			% Benutze Biber/BibLaTeX zum Zitieren
\addbibresource{main.bib}					% Pfad zur BibTeX Datei aus Citavi
\renewcommand{\cite}{\parencite}
\usepackage{caption}
\usepackage{subcaption}
\usepackage{graphicx}
\usepackage{svg}
\usepackage{placeins}
\usepackage[hidelinks]{hyperref}
\usepackage{amsmath}
\usepackage[headsepline]{scrlayer-scrpage}
\clearpairofpagestyles %Seitenzahl nicht in der Kopfzeile
\usepackage{acronym}

\title{MeetEU Project - Team Heidelberg - Team 1 -- \\ Identification and Enhancement of novel Sars-CoV-2 NSP13 helicase inhibitors}
\author{Linda Blaier, Paul Brunner, Selina Ernst, Valerie Segatz and Chlo\'{e} Weiler}
\date{February 2024}

\begin{document}

\maketitle

\ihead{\headmark}
\automark{section}  %Kopfzeile gleich dem Sektiontitel
\cfoot{\pagemark}   %\ofood Seitenzahl rechts

\section{Abstract}
Even though the development of vaccines against Sars-CoV-2 was successful during the recent pandemic, the amount of FDA approved drugs for the therapy of Covid-19 is still limited to Paxlovid and Veklury, Olumiant and Actemra \cite{FDACOVID}. One possibility to accelerate the development of new therapies for Covid19 is to screen already approved drugs for effects against the viral reproduction. In this years MeetEU project, we investigated the NSP13 helicase of Sars-CoV-2 and tried to find compounds that could be repurposed for this therapy, as well as novel compounds that could lead to an effective treatment of Covid19. Using our \textit{in-silico} pipeline (see Figure \ref{workflow}) enables us to evaluate possible drug candidates, suggest novel structures based on already approved drugs and investigate their toxicity, while being cheaper and less labor intensive than projects limited to wet-lab work. 

\begin{figure}[h]
  \centering
  \includegraphics[width=\textwidth]{Workflow_MeetEU.pdf}
  \caption{Proposed workflow for the discovery and improvement of NSP13 helicase inhibitors.}
  \label{workflow}
\end{figure}
\newpage
\FloatBarrier
%    Abkürzungsverzeichnis
{\setlength{\parskip}{0.2cm}
\section*{Abbreviations}
    \begin{acronym}[LC-MS/MS23]
        % A B C D E F G H I J K L M N O P Q R S T U V W X Y Z        
        % Abkürzungen
        \acro{MD}{molecular dynamics}
        \acro{NSP13}{non-structural protein 13}
        \acro{RMSD}{root-mean square deviation}
        \acro{RMSF}{root-mean square fluctuation}
        \acro{MM-PBSA}{molecular mechanics energies combined with the Poisson-Boltzmann and surface area continuum solvation}
        % Formelzeichen
        
        
        % als benutzt markierte Acronyme    
        
        
    \end{acronym}
}
\newpage

\section{Introduction}
Even though the development of vaccines against Sars-CoV-2 was successful during the recent pandemic, the amount of FDA approved drugs for the therapy of Covid-19 is still limited to Paxlovid and Veklury, Olumiant and Actemra \cite{FDACOVID}. As vaccines are not able to prevent 100\% of infections, enhancing the landscape of possible drugs used to treat Covid-19 would especially benefit people at risk of a harsh pathogenesis. The goal of this years Meet-EU project was to investigate the \ac{NSP13} helicase of the Sars-CoV-2 virus for possible vulnerabilities to inhibitors. The helicase is strongly conserved between corona viruses, and thus offers a nice target for drugs, as it is less likely for the drugs to become ineffective due to rapid changes in the viral genome. In general \ac{NSP13} is a vital component of the viral replication mechanism and thus inhibiting it would severely hinder the spread of the virus in the host. The protein consists of five domains, namely the zinc binding domain, the stalk domain, as well as 1A, 2A and 1B. The latter three make up the catalytic centre of the protein, where RNA and ATP bind \cite{NSP13_basics}. \\
In order to explore possible new therapy options it makes sense to perform an \textit{in-silico} screening of different compounds to identify groups of compounds that interact well with the NSP13 helicase. This screening process involves several steps: Identify possible binding sites, screen ligands for how well they bind in this pocket and validate the findings to generate a suggestion, as to which compounds would be worth it to investigate in a wet-lab setting. Focussing this screening on well documented or already FDA approved compounds significantly simplifies a potential registration of a potential new drug entering the market. 
\subsection{Lead Drug Enhancement}
In order to enhance the binding affinity of our drug candidates and thus their performance, we used AutoGrow4 (Version 4.0.3) \cite{packageAutogrow4} to generate novel compounds. Starting with the best binding compounds of our initial docking simulation with AutoDock Vina as generation zero, multiple new structures are generated by combining sub-structures of the first generation or by passing them through a set of possible chemical reactions after converting them into their respective SMILES codes. All of the generated compounds are ranked by their binding affinity. After passing several filters the best performing compounds are used as the seed for the next generation. Using this algorithm, compounds are found, which show higher binding affinities than the first generation. As AutoGrow4 labels all new structures by the path by which they were obtained, we can also evaluate the synthesizability.  

\subsection{Molecular Dynamics Simulation}
As the last step of our pipeline, an \ac{MD} simulation is conducted using the best scoring compounds as a ligand in the binding pocket of the NSP13 protein. Simulating the movement of molecules in a system based on the attracting and repulsing forces between atoms helps us a lot with understanding the molecular interplay between the ligands found and the binding pocket. Whether the ligand stays inside of the binding pocket throughout the whole simulation gives an estimate on how strongly it binds to the protein \cite{MD_Basics}. As this is the final step of our pipeline, the result of this simulation estimates how the ligand would perform if applied \textit{in-vitro}. Additionally, the frames generated by the software can be used to calculate different metrices regarding the binding strength, like the \ac{RMSF}, \ac{RMSD} and \ac{MM-PBSA}. The \ac{RMSD} measures the average displacement of the atoms throughout the simulation compared to the first frame of the simulation and estimates how much the protein moves and changes conformations over time. The \ac{RMSF} on the other hand calculates the movement of a certain atom or group of atoms over time compared to the average position of the atoms and groups \cite{RMSD_RMSF}. \ac{MM-PBSA} is able to estimate the binding affinity of the simulated protein-ligand pair \cite{MM_PBSA}.


\section{Material and Methods}
\subsection{Datasets from ZINC20 and ECBL}
A total of 1616 fda approved drugs were downloaded in .sdf format from the ZINC database \cite{Irwin.2020}. Additional, 5016 fiels were retrieved, downloading the pilot library from the ECBL database.

\subsection{Receptor and ligand preparation}
Ligands were prepared using openbable in order to convert implicit hydrogens into explicit hydrogens, generate necessary 3D structures of the ligands, as well as to split mulitmolecule files into single ones. ADFR suite was further used in order to convert all files into the .pdbqt format, which is required by Autodock Vina. 
 
\subsection{Generation of novel structures}
Using the same PDB file of the nsp13 A-chain, the top 10 best scoring drugs from our initial AutoDock Vina docking were passed to AutoGrow4 (Version 4.0.3 \textcite{packageAutogrow4}) as SMILES. The docker container provided by the package maintainers was used, as it guarantees no problems concerning version incompatibilities. Using a similar configuration to what \Citeauthor{packageAutogrow4} used for lead drug improvement, AutoGrow4 generated and re-scored a plethora of novel structures. Due to our limited amount of structures in generation zero, we opted to disable the generation of cross-over molecules for the first generation. The configuration file can be found in the appendix. As we try to improve a small amount of lead drugs, AutoGrow4 generates many new molecules per generation, while it is only run for five generations. This enables us to explore different modifications to our compounds while not making synthesis too complicated, as there is a maximum of five modifications happening. Because AutoGrow4 also enables structures with a diverse structure to pass into next generations even though they might not be top performers yet, we can be sure to explore a vast landscape of different possible drugs.

\subsection{Molecular Dynamics Simulation}
In order to validate the binding of the discovered compounds, we used GROMACS (Version 2023.3, \textcite{packageGROMACS}) to simulate the drugs inside of the ATP binding site of NSP13. To do so, the Protein-Ligand Complex tutorial by \Citeauthor{Lemkul2018} was followed \cite{Lemkul2018}. The a99SB-disp forcefield was used, as it was shown to recreate protein structures in different environments very accurately \cite{Forcefield}. The 6zsl PDB file was first turned into the fitting GROMACS format (\textit{.gmx}). Using the gmx file, the general simulation parameters are set up. As the ligands feature bonds and atoms not commonly seen in proteins, it is required to create a fitting force-field for them. For this acpype \cite{acpype} was used, which was developed to be compatible with AMBER force-fields. The output files from Maestro Glide were converted to PDB and then piped into acpype. The output of acpype was then combined with the created GROMACS files according to \Citeauthor{Lemkul2018}. Following the tutorial, the simulation space was created, filled with water (a99SBdisp\_water used as its force-field) and ions were added to create a net zero charge system. Energy-minimization was conducted and \textit{NVT} and \textit{NPT} equilibration was used to finish the simulation setup. The production run was set to simulate 100 ns. The \textit{.mdp} file used for the final production run can be seen in the appendix. \\ 
For the analysis of the simulation, GROMACS internal tools were used to calculate the RMSD and RMSF of the protein. These scores give us insight into the movement and conformational changes during the simulation regarding protein and ligand. Furthermore, the simulation frames were extracted and rendered into a video using VMD \cite{VMD}. For MM-PBSA calculations we used g\_mmpbsa \cite{MMPBSA1, MMPBSA2}. This package enables direct calculation of binding affinities using the GROMACS output files.
\section{Results} 

\FloatBarrier

\section{Discussion and Outlook}

\section{Supplementary Material}

\pagebreak
\FloatBarrier

\renewcommand{\bibname}{References}  % damit Literatuverzeicnis mit "References" betitelt
\printbibliography


\end{document}
